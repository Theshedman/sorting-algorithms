\documentclass[a4paper]{report}                                     

\usepackage{ngerman}                                                   
\usepackage[utf8]{inputenc}                                                 
\usepackage{amssymb}                                                                                                 
\usepackage{pseudocode} 
\usepackage{listings}
\usepackage{color}
\usepackage{parskip}
\setlength\parindent{0pt}

\definecolor{dkgreen}{rgb}{0,0.6,0}
\definecolor{gray}{rgb}{0.5,0.5,0.5}
\definecolor{mauve}{rgb}{0.58,0,0.82}



\begin{document}

\chapter*{Radixsort}

\underline{\bf{Laufzeit}}

Da der Aufwand von Radixsort nicht nur von der Anzahl der Elemente $n$, die sortiert werden sollen abhängig ist, sondern außerdem noch von der maximalen Anzahl an Ziffern eines Elements $m$, wird die Laufzeit generell mit $O(m \times n)$ angegeben.

Ist die maximale Anzahl der Ziffern fest bestimmt und somit eine Konstante, so lässt sich die Laufzeit auch durch $O(n)$ beschreiben. 

\underline{\bf{Wann ist dieser Algorithmus sinnvoll einsetzbar?}}

Nehmen wir an, man sortiert Briefe nach der in Deutschland fünfstelligen Postleitzahl, so ist dieser Algorithmus durchaus empfehlenswert, da vermutlich viele Briefe sortiert werden müssen, dies aber in nur 5 Durchläufen geschieht. 

Übersteigt die maximale Anzahl der Ziffern jedoch die Anzahl der Elemente stark, so ist die Anwendung alternativer Algorithmen sinnvoll.


\end{document} 